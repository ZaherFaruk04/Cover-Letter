\documentclass{article}
\usepackage{graphicx} % Required for inserting images

\title{My Interest in Technology}
\author{Zaher Faruk}
\date{April 2025}

\begin{document}

\maketitle

\section{Quantum Computing}

Working on my third-year quantum computing project was an eye-opener into how technology is evolving beyond classical computation. Exploring IBM Quantum, PennyLane, and Microsoft Azure, I got hands-on experience with superconducting qubits, photonic systems, and quantum annealing, learning how quantum algorithms work in real applications. It wasn’t just about theoretical physics—it was about writing code that interacts with real quantum processors, seeing how small optimisations can drastically impact results. The challenge of making sense of quantum mechanics from a programming perspective made me appreciate how much potential this field holds. With technology advancing rapidly, it’s exciting to be part of a generation that’s building the groundwork for something revolutionary.

\section{Engineering}

Engineering my second-year drone project taught me that building something from scratch is one of the best ways to truly understand technology. Designing the flight system in Arduino and modelling parts in Fusion 360 wasn’t just about following instructions—it was about problem-solving on the fly, figuring out why a motor wasn’t working, or why the drone wasn’t balancing properly. Every adjustment taught me something new, whether it was about how hardware interacts with software or how small design choices affect real-world performance. There’s something uniquely satisfying about seeing a project go from an idea to a functioning prototype, and this experience reinforced why hands-on work is such an essential part of learning.

\section{Computational Physics}

Computational physics allows me to turn equations into real-world insights. Using MATLAB, Python, and NumPy, I’ve built models for chaotic systems and atomic interactions, uncovering patterns that aren’t visible through observation alone. These projects have reinforced my appreciation for data-driven problem-solving and the power of simulations in understanding complex systems.

\end{document}
